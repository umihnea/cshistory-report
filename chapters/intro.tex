According to \cite{ucalgary-hom}, \textbf{contemporary multimedia} can be defined as ``the seamless digital integration of text, graphics, animation, audio, still images and motion video in a way that provides individual users with \emph{high levels} of \emph{control} and \emph{interaction}''.

Multimedia is generally made possible by the \emph{digital computer}.
During the personal computer revolution, it has become clear that the media -- now known as the ``old media`` (books, newspapers and magazines) -- will have to adapt in order to integrate digitalization.
The much richer, more enabling mean of the computer would \emph{disrupt} the industry.
The digital revolution lead to the invention of new formats and genres -- video games (a completely new kind of media), software, social media, digital audio, electronic books and others \cite{wiki:Digital_media}.

Combined with the development and popularity of the World Wide Web, and propulsed further by the improvements in data compression, multimedia has led \emph{the digital revolution}.

Video is a crucial part of the spectrum, as moving images have long captivated the human race.
Video storage and video streaming over the Internet has changed our lives radically in the past two decades.
In the report below, we treat the subject of digital motion video from its inception to today's Internet.