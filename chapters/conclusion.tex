Video technology has evolved significantly -- since the first algorithm to effectively use DCT for video compression back in the 1980s, to one of the most important elements at the core of the modern Internet today.

We have seen how a cosine wave decomposition helped advance video compression, along with motion compensation.
Shortly after the invention of motion-compensated DCT, the first official standards appeared, backed by large organizations (among which are the MPEG and the ITU-T, which played a key role).
The MPEG and H.26x formalized the standards and built the codecs fundamental to efficiently storing and streaming videos in digital mediums.
Their efforts continued into the 2000s, with HEVC being their latest standard -- HEVC greatly extends the capabilities of its predecessors.
Finally, we have seen how new media businesses owe some of their success to this progress.

The development of new video compression techniques, codecs and standards continues \textbf{today}.
For a brief period in recent history, a wave of proprietary spin-offs dominated the scene and large companies filed patents to protect their independently-developed technologies.
The lack of a unitary, open video codec standard that could benefit from cooperation meant that progress was slowed down.
This ``licence war'' has been addressed today by introducing new open standards, backed by the giants of the Web, forming the \emph{Alliance for Open Media} or \emph{AOMedia} for short \cite{wiki:Alliance_for_Open_Media}.
They developed \emph{AV1}, an ``open, royalty-free coding format designed for video transmission over the Internet'' \cite{wiki:AV1}.
This marks the beginning of a new open future for multimedia codecs.
